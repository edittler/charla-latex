\documentclass[svgnames]{beamer}

\usepackage[utf8]{inputenc} % Indica cuál es la codificación de este archivo
\usepackage[spanish]{babel} % Indica el idioma en que está escrito el documento
\usepackage{amssymb} % Fuentes y símbolos adicionales (por AMS)
\usepackage{amsmath} % Mejoras a entornos matemáticos y extras (por AMS)
\usepackage{mathtools} % Correcciones a amsmath y funcionalidades extras
\usepackage{listings} % Permite ingresar código fuente de software

\title{Cómo hacer informes con \LaTeX}
\subtitle{Y subir la nota de los trabajos prácticos}
\author{Ezequiel ``Pampa'' Pérez Dittler}
\date{Semana LUGFI, Agosto de 2016}

\begin{document}

\frame{\titlepage}

\begin{frame}
	\frametitle{¿Qué es LUGFI?}
	
	Es el grupo de usuarios y desarrolladores de software libre y abierto de la Facultad de Ingeniería de la UBA
	
	Acrónimo recursivo:
	
	LUGFI Usa GNU/Linux en la Facultad de Ingeniería
\end{frame}

\begin{frame}
	\frametitle{¿Qué es \LaTeX?}
	\begin{itemize}
		\item Es un sistema de composición de textos
		\item Orientado a la creación de documentos con alta calidad tipográfica
		\item Muy popular en el entorno académico, especialmente entre matemáticos, físicos, químicos e informáticos
		\item Es un conjunto de macros de \TeX
	\end{itemize}
\end{frame}

\end{document}
