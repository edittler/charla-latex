\documentclass{article}
\usepackage[utf8]{inputenc}
\usepackage[spanish]{babel}
\selectlanguage{spanish}
\usepackage{multirow}
\usepackage{longtable}
\usepackage{listings}
\setcounter{secnumdepth}{0}
% Page layout (geometry)
\usepackage[a4paper,headheight=0cm,scale={0.7,0.8},hoffset=0.cm,left=2cm,right=2cm,top=2cm, bottom = 3cm]{geometry}

%seccion de definiciones
\newcommand{\subsubsubsection}[1]{ \textbf{#1} \\}

\newcommand{\tabladeconstantes}[4]{
 \renewcommand{\tabcolsep}{0.3em}
 \begin{longtable}{ccc}
  \hline
  $ID_{int}$ & $ID_{enum}$ & $ID_{ext}$ & Valor \\
  \hline
 \endfirsthead
  #1
  \hline
  $ID_{int}$ & $ID_{ext}$ & Valor \\
  \hline
 \endhead
  \hline
  #2
  \hline
 \caption{Tabla de Registros#3}
 \label{#4}
 \end{longtable}
}

\newcommand{\tabladeenumerados}[4]{
 \renewcommand{\tabcolsep}{0.3em}
 \begin{longtable}{|c|c|c|c|c|}
  \hline
  $ID_{int}$ & $ID_{ext}$ & $L_i/L_s$ & Base & Tamaño \\
  \hline
 \endfirsthead
  #1
  \hline
  $ID_{int}$ & $ID_{ext}$ & $L_i/L_s$ & Base & Tamaño \\
  \hline
 \endhead
  \hline
  #2
  \hline
 \caption{Tabla de Enumerados#3}
 \label{#4}
 \end{longtable}
}

\newcommand{\tabladearreglos}[4]{
 \renewcommand{\tabcolsep}{0.3em}
 \begin{longtable}{|c|c|c|c|c|c|c|}
  \hline
  $ID_{int}$ & $ID_{ext}$ & $L_i/L_s$ & Elementos & Base & Tamaño & Dimensión \\
  \hline
 \endfirsthead
  #1
  \hline
  $ID_{int}$ & $ID_{ext}$ & $L_i/L_s$ & Elementos & Base & Tamaño & Dimensión \\
  \hline
 \endhead
  \hline
  #2
  \hline
 \caption{Tabla de Arreglos#3}
 \label{#4}
 \end{longtable}
}

\newcommand{\tabladeregistros}[4]{
 \renewcommand{\tabcolsep}{0.3em}
 \begin{longtable}{|c|c|c|c|c|c|c|}
  \hline
  $ID_{int}$ & $ID_{ext}$ & $ID_{dat}$ & Base & Tamaño & Offset & Total \\
  \hline
 \endfirsthead
  #1
  \hline
  $ID_{int}$ & $ID_{ext}$ & $ID_{dat}$ & Base & Tamaño & Offset & Total \\
  \hline
 \endhead
  \hline
  #2
  \hline
 \caption{Tabla de Registros#3}
 \label{#4}
 \end{longtable}
}

\newcommand{\tabladevariables}[4]{
 \renewcommand{\tabcolsep}{0.3em}
 \begin{longtable}{|c|c|c|c|c|}
  \hline
  $ID_{int}$ & $ID_{ext}$ & Base & Tamaño & Offset \\
  \hline
 \endfirsthead
  #1
  \hline
  $ID_{int}$ & $ID_{ext}$ & Base & Tamaño & Offset \\
  \hline
 \endhead
  \hline
  #2
  \hline
 \caption{Tabla de variables#3}
 \label{#4}
 \end{longtable}
}
\title{Parcial Teórico}
\date{2015-10-29}
\pagestyle{empty}


\begin{document}

\section{Algoritmos y Programacion II -- Catedra Lic. Gustavo Carolo \\ Evaluacion Parcial -- 2015-10-29}
\renewcommand{\arraystretch}{1.5}
\noindent\begin{tabular}{p{7.9cm}@{}p{2cm}r}
\multicolumn{2}{c}{-- Entregar teoría y práctica por separado --} & \multirow{3}{*}{\begin{tabular}{|c|c|c|c|c|c|c|c|c||c|}
\hline
a & b & c & d & e & f & g & h & i & T\\
\hline
% & & \backslashbox{}{} & & & & & \\
 & & & & & & & & & \\
\hline
\end{tabular}} \\
\multicolumn{2}{l}{Nombre: \dotfill {} }\\
Mail: \dotfill {} Padrón: & \dotfill \\
\end{tabular}

\subsection{Teor\'ia Tema 1}

\renewcommand{\baselinestretch}{0.9}

Dado el siguiente código:

\lstset{language=C, basicstyle=\scriptsize\sl, numbers=left,tabsize=2, stepnumber=5, numbersep=5pt, keywordstyle=\scriptsize\bf,identifierstyle=\scriptsize\tt, xleftmargin=2cm,basewidth={0.57em,0.57em}}
\lstinputlisting{ptt1-2015-10-29.c}

Asuma que:
\begin{itemize}
  \setlength{\itemsep}{0pt}
  \setlength{\parskip}{0pt}
  \setlength{\parsep}{0pt}
  %\item {Se está compilando en una arquitectura de 32 bits (4 bytes) por palabra.}
 %\item {Se está compilando en una arquitectura de 128 bits (16 bytes) por palabra.}
 \item {Se está compilando en una arquitectura de 64 bits (8 bytes) por palabra.}
 %\item {Un {\bf char} ocupa 8 bits; un {\bf short} ocupa 16 bits; un {\bf double} ocupa 64 bits; el resto de los tipos elementales ocupa 32 bits.}
 \item {Un {\bf char} ocupa 8 bits; un {\bf short} ocupa 16 bits; un {\bf float} ocupa 32 bits; el resto de los tipos elementales ocupa 64 bits.}
 %\item {Un {\bf char} ocupa 8 bits; un {\bf short} ocupa 16 bits; un {\bf float} ocupa 64 bits; el resto de los tipos elementales ocupa 128 bits.}
 \item {El compilador ordena las variables en el orden dado y no las serializa (deja espacios en desuso).}
 \item {Las variables \verb!argc! y \verb!argv! en \verb!main! no son necesarias.}
 \item {La memoria siempre se aloca y \verb!malloc! nunca devuelve \verb!NULL!.}
\end{itemize}

Se pide:

\renewcommand\theenumi{\alph{enumi}}
\renewcommand\theenumii{\roman{enumii}}
\begin{enumerate}
 \setlength{\itemsep}{1pt}
 \item {Utilice un diagrama adecuado para justificar el seguimiento recursivo de la función \verb!func! e indique la salida que imprime.}
 \item {¿Cuál es el orden de recursividad de la función \verb!func!? ¿es directa o indirecta? ¿Cuál es la condición de corte?}
 \item {Divida con barras los distintos lexemas la función \verb!func! de la misma forma en que lo haría el {\it scanner}. Indique cual es la cadena que no es procesada como tal por el {\it scanner}.}
 \item {El analizador sintáctico define un operador ternario (es decir que recibe tres parámetros), ¿cómo se utliza? ¿conoce operadores unarios?}
 \item {Complete en bytes las tablas del compilador y la de variables para la función \verb!main!. Ignore las tablas de constantes y enumerados.}
 \item {Responda cuántos bytes habría entre \verb!a->attributes[10].value[-67]! y \verb!a->number.id_jedi!. Justifique la cuenta. No analice violaciones de segmentos.}
 \item {Muestre en un diagrama como quedaran los punteros en memoria. Indique a que parte de la memoria
pertenece cada bloque (recuerde incluir todas las variables, incluso aquellas que no se utilicen; \verb!argv! y
\verb!argc! pueden ser ignoradas). Indique todos los datos que no fueron inicializados. Muestre el resultado que sale al ejecutarse la impresión del \verb!printf! de la línea 45.}
 \item {¿Es posible liberar toda la memoria en el caso anterior? Escriba el código que intente liberar la mayor cantidad posible de memoria (o toda si contesto afirmativamente a la pregunta anterior).}
 \item {Indicar si en el caso anterior las siguientes expresiones compilan y en caso de hacerlo, un identificador interno o externo tipo:
  \begin{enumerate}
   \item {\verb!*((*a).next->firstname)!}
   \item {\verb!&((*b).number)!}
   \item {\verb!*(c->prior-7)!}
  \end{enumerate}
 }
\end{enumerate}

\renewcommand{\arraystretch}{1}
{ \footnotesize
%\tabladeenumerados{}{
%&&&&\\
%\hline
%&&&&\\
%}{}{}
\tabladevariables{}{
&&&&\\
\hline
&&&&\\
\hline
&&&&\\
\hline
&&&&\\
}{}{}
\tabladearreglos{}{
&&&&&&\\
\hline
&&&&&&\\
}{}{}
\tabladeregistros{}{
&&&&&&\\
\cline{3-6}
&&&&&&\\
\cline{3-6}
&&&&&&\\
\cline{3-6}
&&&&&&\\
\cline{3-6}
&&&&&&\\
\cline{3-6}
&&&&&&\\
\cline{3-6}
&&&&&&\\
\cline{3-6}
&&&&&&\\
\cline{3-6}
&&&&&&\\
\cline{3-6}
&&&&&&\\
\cline{3-6}
&&&&&&\\
\cline{3-6}
&&&&&&\\
\cline{3-6}
&&&&&&\\
\cline{3-6}
&&&&&&\\
\cline{3-6}
&&&&&&\\
\cline{3-6}
&&&&&&\\
}{}{}
}


\end{document}

