\documentclass[a4paper]{article}
\usepackage[utf8]{inputenc} % para poder usar tildes en archivos UTF-8
\usepackage[spanish]{babel} % para que comandos como \today den el resultado en castellano
\usepackage{fullpage}
\usepackage{amsmath}
\usepackage{amssymb}
\usepackage[usenames,dvipsnames,svgnames,table]{xcolor} % Colores
\usepackage[pdftex]{graphicx} % Permite incluir gráficos
\usepackage[hang, bf]{caption} % Personaliza los subtítulos de las figuras y tablas
\usepackage{float} % Permite posicionar mejor las figuras y tablas
\usepackage{listings} % Permite insertar código fuente
\usepackage{pgfplots}
\pgfplotsset{compat=1.5.1}

\definecolor{darkblue}{rgb}{0,0,0.4}
\definecolor{darkgreen}{rgb}{0,0.4,0}

\lstset{ % Elijo formato de bloques de código fuente
  backgroundcolor=\color{gray!5},
  basicstyle=\ttfamily\footnotesize,
  breaklines=true,
  commentstyle=\color{darkgreen},
  extendedchars=true,
  frame=single,
  language=octave,
  literate={á}{{\'a}}1 {é}{{\'e}}1 {í}{{\'i}}1 {ó}{{\'o}}1 {ú}{{\'u}}1 {ñ}{{\~n}}1, % Escapeo caracteres especiales
  keywordstyle=\color{darkblue},
  numbers=left,
  numberstyle=\tiny\color{gray},
  tabsize=4,
  showspaces=false,
  showstringspaces=false,
  stringstyle=\color{red}
}

\begin{document}

\begin{titlepage}
  \centering
  \includegraphics[width=0.6\textwidth]{logo_fiuba.pdf}\par\vspace{1cm}
  {\scshape\LARGE Universidad de Buenos Aires \par}
  {\scshape\LARGE Facultad de Ingeniería \par}
  \vspace{1cm}
  {\scshape\Large 61.09 Probabilidad y Estadística B\par}
  {\scshape\Large Curso 27\par}
  \vspace{1.5cm}
  {\huge\bfseries Método Monte Carlo\par}
  \vspace{2cm}
  {\Large\itshape Nombre y Apellido \par}
  {\Large\itshape Padrón 100.000 \par}

  \vfill

  % Bottom of the page
  {\large \today\par}
\end{titlepage}

\newpage

\section*{Enunciado}
Sea $f: \mathbb{R} \to \mathbb{R}$ una función continua y no negativa.
Sea $M > 0$ el valor máximo de la función $f$ sobre el intervalo $[a, b]$. \\

(a) Se elige al azar un punto de coordenadas $(X, Y)$ dentro del rectángulo de
vértices $(a, 0)$, $(b, 0)$, $(b, M)$, $(a, M)$. Relacionar la probabilidad
del evento $A = \{ Y \leqslant f(X) \}$ con el valor de la integral
$\int\limits_a^b f(x) dx$. \\

(b) Si se conoce el valor de la probabilidad, $P(A)$, del evento
$A = \{ Y \leqslant f(X) \}$, ¿cómo se calcula la integral
$\int\limits_a^b f(x) dx$? \\

(c) Obtener un método que permita estimar el valor de la integral
$\int\limits_a^b f(x) dx$ en base a los resultados de $n$ simulaciones del
experimento descrito en Inciso (a). \\

(d) Estimar el valor de la integral $\int\limits_0^2 e^{-x^{2}} dx$ utilizando
el método obtenido en Inciso (c) basándose en los resultados de 10000
simulaciones.

\newpage
\section*{Resolución}
\subsection*{Inciso (a)}
\begin{figure}[h]
  \centering
  \begin{tikzpicture}
    \begin{axis}[
        xlabel={$X$},
        ylabel={$Y$},
        axis x line=bottom,
        axis y line=left,
        enlarge x limits=true,
        enlarge y limits=true,
        xmin=0.18, xmax=2,
        ymin=0.09, ymax=1,
        %symbolic x coords={a,b},
        area style,
        legend cell align=left,
        legend pos=outer north east
      ]
      \addplot coordinates {(0,1) (2,1)} \closedcycle;
      \addplot {e^(-x^2)};
      \legend{$\Omega$,$A$}
    \end{axis}
  \end{tikzpicture}
  %\caption{Rectángulo de vértices $(a, 0)$, $(b, 0)$, $(b, M)$, $(a, M)$ con $a=0$, $b=2$, $M=1$}
\end{figure}

\[
  P(A) = P( Y \leqslant f(X) ) = \frac{ area(A) }{ area(\Omega) } =
  \frac{ \int\limits_a^b f(x) dx }{ (b-a) . M }
\]

\subsection*{Inciso (b)}

Siendo $P(A)$ conocido, se puede estimar el valor de la integral
$\int\limits_a^b f(x) dx$ del siguiente modo

\[
  \int\limits_a^b f(x) dx = P(A).(b-a) . M
\]

\subsection*{Inciso (c)}
Siendo conocidos los valores de $a$, $b$, $M$ y conocida la función $f(x)$,
se puede estimar el valor de la integral $\int\limits_a^b f(x) dx$ obteniendo
un valor estimado de $P(A)$, $A = \{ Y \leqslant f(X) \}$. \\

Para ello, se genera un vector aleatorio $(X,Y)$ y se verifica si forma
parte del evento $A$, es decir, si $Y \leqslant f(X)$. Repitiendo varias veces
la simulación se puede obtener un mejor valor de $P(A)$. \\

Siendo $n$ la cantidad de simulaciones, la probabilidad se puede estimar de la 
siguiente forma:

\[
  P(A) = \frac{ \#\{
      i \, \epsilon \{ 1, \dots, n \} : y_i \leqslant f(x_i) \}
    }{ n }
\]

\newpage
\subsection*{Inciso (d)}
Se desea el valor de $\int\limits_0^2 e^{-x^2} dx$ utilizando el método,
basándose en resultados de 10000 simulaciones.
De la integral se pueden extraer los siguientes datos:

\begin{itemize}
\item $a = 0$
\item $b = 2$
\item $M = 1$
\item $f(x) = e^{-x^2}$
\end{itemize}
~\\

Se diseñó el siguiente algoritmo en Octave:

\lstinputlisting[
    caption={Algoritmo Monte Carlo en Octave},
    label={lst:monte_carlo}
  ]{../src/monte_carlo.m}
~\\

Corriendo el algoritmo, se obtuvo el valor 0.88460.
Siendo el valor analítico de la integral 0.88208, se puede concluir que el
método Monte Carlo aproxima bastante al valor analítico.

\end{document}
