\documentclass[svgnames]{beamer}
\usetheme{JuanLesPins} % descomentar para cambiar el tema
%\usecolortheme{beaver}

\usepackage[utf8]{inputenc} % Indica cuál es la codificación de este archivo
\usepackage[spanish]{babel} % Indica el idioma en que está escrito el documento
\usepackage{graphicx}
\usepackage{xcolor}
\usepackage{pgfplots}
\usepackage{tikz}
\definecolor{bblue}{HTML}{4F81BD}
\definecolor{rred}{HTML}{C0504D}
\definecolor{ggreen}{HTML}{9BBB59}
\definecolor{ppurple}{HTML}{9F4C7C}
\usepackage{amssymb} % Fuentes y símbolos adicionales (por AMS)
\usepackage{amsmath} % Mejoras a entornos matemáticos y extras (por AMS)
\usepackage{mathtools} % Correcciones a amsmath y funcionalidades extras

\usepackage{listings} % Permite ingresar código fuente de software

\lstset{ % Defino el formato de bloques de código fuente
  inputencoding=utf8, % Indico la codificación de los archivos de entrada
  %extendedchars=true, % Extiendo los caracteres
  % Escapeo caracteres especiales
  literate={`}{{\`{}}}1 {¡}{{!`}}1 {¿}{{?`}}1 {á}{{\'a}}1 {é}{{\'e}}1 {í}{{\'i}}1 {ó}{{\'o}}1 {ú}{{\'u}}1 {ñ}{{\~n}}1,
  showstringspaces=false, % Indica si muestra los espacios dentro de strings
  numbers=left, % Posición en que se muestran los números de línea
  numberstyle=\tiny\color{gray}, % Estilo de los números de línea
  breaklines=true, % Se parten las líneas que exceden del ancho del documento
  frame=single, % Formato del marco del entorno
  backgroundcolor=\color{gray!5}, % Color de fondo
  basicstyle=\ttfamily\scriptsize, % Estilo de base (familia, tamaño, color)
}

\lstdefinestyle{latex}{
  tabsize=4,
  language=[LaTeX]TeX,
  keywordstyle=\color{DarkGreen}, % Estilo de las palabras reservadas
  stringstyle=\color{DarkBlue}, % Estilo de los strings
  commentstyle=\color{DarkGray}, % Estilo de los comentarios
}

\lstdefinestyle{md}{
  numbers=none
}

\usepackage{hyperref}

% Beamer settings

\beamertemplatenavigationsymbolsempty

\AtBeginSection[]{
  \begin{frame}
    \vfill
    \centering
    \begin{beamercolorbox}[
      sep=8pt,
      center,
      shadow=true,
      rounded=true
    ]{title}
      \usebeamerfont{title}\insertsectionhead\par%
    \end{beamercolorbox}
    \vfill
  \end{frame}
}

\AtBeginSubsection[]{
  \begin{frame}
    \vfill
    \centering
    \begin{beamercolorbox}[
      sep=8pt,
      center,
      shadow=true,
      rounded=true
    ]{title}
      %\usebeamerfont{title}\insertsectionhead\par%
      \usebeamerfont{subtitle}\insertsubsectionhead\par%
    \end{beamercolorbox}
    \vfill
  \end{frame}
}

% Document Properties

\title{Herramientas libres para crear documentos de alta calidad}
\subtitle{Algo de {\LaTeX} y Markdown}
\author{Ezequiel Pérez Dittler \and Lucas Perea}
\date{Semana LUGFI, Agosto de 2016}

\begin{document}

\frame{\titlepage}

\section{LUGFI}

\begin{frame}
  \frametitle{¿Qué es LUGFI?}

  \centering

  Es el grupo de usuarios y desarrolladores de software libre y abierto de la Facultad de Ingeniería de la UBA

  \vfill

  \textbf{Acrónimo recursivo}

  \medskip

  LUGFI Usa GNU/Linux en la Facultad de Ingeniería
\end{frame}

\section{\LaTeX}

\begin{frame}
  \frametitle{¿Qué es \LaTeX?}
  \begin{itemize}
    \item Es un sistema de composición de textos
    \item Orientado a la creación de documentos con alta calidad tipográfica
    \item Muy popular en el entorno académico, especialmente entre matemáticos, físicos, químicos e informáticos
    \item Es un conjunto de macros de \TeX
    \item Es un lenguaje de maquetación (similar a HTML)
  \end{itemize}
\end{frame}

\begin{frame}
  \frametitle{¿Por qué usar \LaTeX?}
  \begin{itemize}
    \item Control flexible del contenido
    \item Fórmulas matemáticas de alta calidad
    \item Permite centrarse en el contenido, no en la presentación
    \item Extensible con paquetes
    \begin{itemize}
      \item La presentación se puede cambiar con plantillas
    \end{itemize}
  \end{itemize}
\end{frame}

\begin{frame}
  \frametitle{¿Para qué usar \LaTeX?}
  \begin{itemize}
    \item Trabajos prácticos
    \item Documentos con fórmulas matemáticas
    \item Documentación de código o software
    \item Tesis, artículos científicos
    \item Currículum Vitae
    \item Presentaciones
    \item Partituras musicales
  \end{itemize}
\end{frame}

\begin{frame}
  \frametitle{¿Cómo usar \LaTeX?}
  \framesubtitle{En forma local}
  Distribución \LaTeX
  \begin{itemize}
    \item Linux: TeX Live
    \item Windows: MiKTeX
    \item OSX (Mac): MacTex
  \end{itemize}

  \vfill

  Editor de texto
  \begin{itemize}
    \item Editor de texto plano favorito (por ejemplo, Bloc de notas)
    \item TeXStudio
    \item Texmaker
    \item LyX
  \end{itemize}
\end{frame}

\begin{frame}
  \frametitle{¿Cómo usar \LaTeX?}
  \framesubtitle{En línea}
  No requiere instalar nada, sólo se necesita un navegador de internet.
  \begin{itemize}
    \item Overleaf: www.overleaf.com
    \item ShareLaTeX: www.sharelatex.com
  \end{itemize}
\end{frame}

\subsection{Conceptos básicos}
\begin{frame}
  \frametitle{¡Hola Mundo!}
  \lstinputlisting[style=latex]{hola-mundo/hola-mundo.tex}
\end{frame}

\begin{frame}
  \frametitle{Estructura del documento}
  \begin{columns}
    \column{0.45\textwidth}\centering
      \lstinputlisting[style=latex]{estructura/estructura.tex}
    \column{0.55\textwidth}\centering
      \begin{itemize}
        \item Preámbulo
        \begin{itemize}
          \item Inclusión y configuración de paquetes
          \item Configuración del documento
          \item Propiedades del documento
        \end{itemize}
        \item Contenido del documento
        \item Fin del documento
        \begin{itemize}
          \item Todo lo que se inserte aquí no tendrá efecto alguno tanto en la configuración como en el contenido
        \end{itemize}
      \end{itemize}
  \end{columns}
\end{frame}

\begin{frame}
  \frametitle{La codificación del caracteres del siglo XXI es... UTF-8!}
  \begin{columns}
    \column{0.45\textwidth}\centering
      \lstinputlisting[style=latex]{hola-mundo-utf/hola-mundo.tex}
    \column{0.55\textwidth}\centering
      \begin{itemize}
        \item Se incluyó el paquete \texttt{inputenc}
        \item Permite especificar la codificación del archivo fuente
        \item La opción \texttt{utf8} indica que la codificación es UTF-8
      \end{itemize}
  \end{columns}
\end{frame}

\begin{frame}
  \frametitle{¿La fuente debe ser de 12pt? ¿Hoja A4? ¿Tipo libro?}
  \centering
  \lstinputlisting[style=latex]{tamanios/hola-mundo.tex}
\end{frame}

\begin{frame}
  \frametitle{Más opciones para personalizar}
  \begin{columns}
    \column{0.45\textwidth}\centering
      Clases\\[.2cm]
      \begin{itemize}
        \item article
        \item book
        \item beamer
        \item IEEEtran
        \item exam
        \item y varios más...
      \end{itemize}
    \column{0.55\textwidth}\centering
      Opciones\\[.2cm]
      \begin{itemize}
        \item 10pt,11pt,12pt,. . .
        \item a4paper,letterpaper,b5paper,. . .
        \item twoside, oneside
        \item landscape
        \item onecolumn, twocolumn
        \item draft
        \item y varios más...
      \end{itemize}
  \end{columns}
\end{frame}

\begin{frame}
  \frametitle{Ordenando en secciones}
  \lstinputlisting[style=latex]{secciones/secciones.tex}
\end{frame}

\begin{frame}
  \frametitle{Agregando índices}
  \begin{itemize}
    \item \texttt{\textbackslash{}tableofcontents}
    \item \texttt{\textbackslash{}listoffigures}
    \item \texttt{\textbackslash{}listoftables}
    \item y algunos más...
  \end{itemize}
\end{frame}

\begin{frame}
  \frametitle{¿Que tal una carátula?}
  \lstinputlisting[style=latex]{caratula/hola-mundo.tex}
\end{frame}

\begin{frame}[fragile]
  \frametitle{¡Los títulos de índices y las fechas se ven en inglés!}
  \framesubtitle{El paquete \texttt{babel} viene al rescate}

  \begin{lstlisting}[style=latex]
\usepackage[spanish]{babel}
  \end{lstlisting}

  \begin{itemize}
    \item El paquete \texttt{babel} traduce titulos, fechas
    \item Habilita la separación silábica en el idioma indicado
    \item La opción \texttt{spanish} indica que el idioma del documento es el español
    \item Para usar \texttt{babel} o indicar idiomas puede requerir instalar paquetes en la distribución \LaTeX
  \end{itemize}
\end{frame}

\subsection{Fórmulas matemáticas}

\begin{frame}
  \frametitle{Escribiendo fórmulas matemáticas}
  \begin{itemize}
    \item Sintáxis específica
    \item Sólo hay que indicar que vamos a escribir fórmulas
    \begin{itemize}
      \item inline (junto con texto)
      \item en una nueva línea
    \end{itemize}
   \end{itemize}
\end{frame}

\begin{frame}
  \frametitle{Ejemplo}
  \only<1> {\lstinputlisting[style=latex]{matematicas/matematicas.tex}}
  \only<2> {Sean $ \alpha, \beta \in \mathbb{R} $\begin{equation*} \alpha + \beta = \beta + \alpha\end{equation*} }
\end{frame}

\begin{frame}
  \frametitle{Más ejemplos}
  Sumatorias
  {\lstinputlisting[style=latex]{matematicas/sumatoria.tex}}
  \begin{center}
    $\sum_{i=0}^{n} a_ix^i$
  \end{center}\pause
  Integrales
  {\lstinputlisting[style=latex]{matematicas/integral.tex}}
  \begin{center}
    $\int_{a}^{b} x^2 dx$
  \end{center}\pause
  Límites
  {\lstinputlisting[style=latex]{matematicas/limite.tex}}
  \begin{center}
    $\lim_{x\to\infty} f(x)$
  \end{center}
\end{frame}

\subsection{Gráficos}

\begin{frame}[fragile]
  \frametitle{Importar gráficos}
  \begin{columns}
    \column{0.5\textwidth}\centering
      \lstinputlisting[style=latex]{graficos/documento-include.tex}
    \column{0.5\textwidth}\centering
      \begin{itemize}
        \item Se incluyó el paquete \texttt{graphicx}
        \item El entorno \texttt{figura}
        \begin{itemize}
            \item Se importa la imagen con \texttt{includegraphics}
            \item Opcionalmente se puede agregar un subtítulo (\texttt{caption})
            \item Opcionalmente se puede agregar una etiqueta de referencia (\texttt{label})
        \end{itemize}
      \end{itemize}
  \end{columns}
\end{frame}

\begin{frame}[fragile]
  \frametitle{Importar gráficos}
  \framesubtitle{Ejemplo}
  \lstinputlisting[style=latex]{graficos/include.tex}
  \begin{figure}
    \includegraphics[width=0.3\textwidth]{graficos/logo_uba}
    \caption{Logo UBA}
    \label{fig:logo_uba}
  \end{figure}
\end{frame}

\begin{frame}[fragile]
  \frametitle{Tikz}
  \framesubtitle{Gráfico de funciones}
    \begin{center}
      \begin{tikzpicture}[domain=0:4]
        \draw[very thin,color=gray] (-0.1,-1.1) grid (3.9,3.9);
        \draw[->] (-0.2,0) -- (4.2,0) node[right] {$x$};
        \draw[->] (0,-1.2) -- (0,4.2) node[above] {$f(x)$};
        \draw[color=red]
        plot (\x,\x)
        node[right] {$f(x) =x$};
        % \x r means to convert ’\x’ from degrees to _r_adians:
        \draw[color=blue]
        plot (\x,{sin(\x r)})
        node[right] {$f(x) = \sin x$};
        \draw[color=orange] plot (\x,{0.05*exp(\x)}) node[right] {$f(x) = \frac{1}{20} \mathrm e^x$};
      \end{tikzpicture}
    \end{center}
\end{frame}

%\begin{frame}
%  \frametitle{PGF Plots}
%  \framesubtitle{Gráfico de barras}
%  \begin{tikzpicture}
%    \begin{axis}[
%      x tick label style={/pgf/number format/1000 sep=},
%      ylabel=Population,
%      enlargelimits=0.1,
%      legend style={
%        at={(0.5,-0.15)},
%        anchor=north,
%        legend columns=-1},
%      ybar,
%      bar width=7pt,
%    ]
%      \addplot coordinates{(1930,50e5) (1940,33e5) (1950,40e5) (1960,50e5) (1970,70e5)};
%      \addplot coordinates {(1930,38e5) (1940,42e5) (1950,43e5) (1960,45e5) (1970,65e5)};
%      \addplot coordinates {(1930,15e5) (1940,12e5) (1950,13e5) (1960,25e5) (1970,35e5)};
%      \addplot[
%        red,
%        line legend,
%        sharp plot,
%        update limits=false
%      ] coordinates {(1910,4.3e6) (1990,4.3e6)}
%      node[above] at (1950,4.3e6) {Houses};
%      \legend {Far,Near,Here,Annot}
%    \end{axis}
%  \end{tikzpicture}
%\end{frame}

\subsection{Código}

\begin{frame}
  \frametitle{Escribir código}
  En un procesador de texto: 
  \begin{itemize}
    \item \texttt{ctrl+C ctrl+V} y cruzamos los dedos
    \item Después damos formato
  \end{itemize}\pause
  En \LaTeX:
  \begin{itemize}
    \item Formato agradable a la vista
    \begin{itemize}
      \item Syntax highlighting
      \item Personalización de estilos (números de líneas, colores)
    \end{itemize}
    \item ¡Posibilidad de importar código fuente desde otros archivos!
  \end{itemize}
\end{frame}

\begin{frame}
  \frametitle{Incluyendo código en el documento}
  \begin{columns}
    \column{0.55\textwidth}\centering
      \lstinputlisting[style=latex, firstline=1, lastline=9]{codigo/importar-listings.tex}
    \column{0.45\textwidth}\centering
      \begin{itemize}
        \item Hay que importar el paquete \texttt{listings}
        \item Se puede personalizar el estilo del código (importar \texttt{color})
        \item También el tamaño de los tabs
        \item Y muchas cosas más...
      \end{itemize}
  \end{columns}
\end{frame}


\begin{frame}[fragile]
  \frametitle{Incluyendo código en el documento}
  \framesubtitle{Uso básico}
  \lstinputlisting[style=latex, firstline=13, lastline=15]{codigo/importar-listings.tex}

  \vfill

  Da como resultado:
  \begin{lstlisting}
  print("Hola mundo!")
  \end{lstlisting}\pause

  \vfill

  \begin{center}
    \emph{¡Lo estuvimos usando todo el tiempo!}
  \end{center}
\end{frame}

\begin{frame}[fragile]
  \frametitle{Incluyendo código en el documento}
  \framesubtitle{Personalizando un poco más}
  \lstinputlisting[style=latex, firstline=17, lastline=23]{codigo/importar-listings.tex}

  \vfill

  Da como resultado:
  \begin{lstlisting}[language=python]
  def main():
    """Saluda al mundo"""
    print("Hola mundo")
  
  main()
  \end{lstlisting}
\end{frame}

\begin{frame}[fragile]
  \frametitle{Incluyendo código en el documento}
  \framesubtitle{Importando un archivo con código}
  Indicamos el nombre del archivo a mostrar:
  \lstinputlisting[style=latex, firstline=25, lastline=25]{codigo/importar-listings.tex}\pause

  \vfill

  Podemos incluir sólo algunas líneas:
  \lstinputlisting[style=latex, firstline=26, lastline=26]{codigo/importar-listings.tex}

\end{frame}

\section{Markdown}

\subsection{Introducción}

\begin{frame}
  \frametitle{¿Qué es Markdown?}
  \begin{itemize}
    \item Es un lenguaje de marcado ligero
    \item Inspirado en convenciones existentes marcar mensajes de correo electrónico usando texto plano
    \item El texto markdown se convierte a HTML en las herramientas más populares
  \end{itemize}\pause
\end{frame}

\begin{frame}
  \frametitle{¿Por qué usar Markdown?}
  \begin{itemize}
    \item El marcado es sencillo de aprender y usar
    \item Los caracteres de marcado son poco intrusivo
    \item Es utilizado en muchas plataformas
    \begin{itemize}
      \item GitHub, GitLab
      \item Foros
      \item Mensajería instantánea (por ejemplo, Slack)
      \item Aplicaciones de notas
    \end{itemize}
  \end{itemize}
\end{frame}

\begin{frame}
  \frametitle{¿Cómo usar Markdown?}
  Aplicaciones de escritorio:
  \begin{itemize}
    \item Editor de texto plano favorito
    \item Editor Atom (incluye previsualización)
    \item Otros editores con plugin de previsualización
  \end{itemize}
  \vfill
  Servicios en línea:
  \begin{itemize}
    \item Dillinger (dillinger.io)
    \item Multitud de editores disponible
  \end{itemize}
\end{frame}

\subsection{Sintaxis básica}

\begin{frame}[fragile]
  \frametitle{Encabezados}
  \begin{lstlisting}[style=md]
# Encabezado 1
## Encabezado 2
### Encabezado 3
#### Encabezado 4
##### Encabezado 5
###### Encabezado 6
  \end{lstlisting}
\end{frame}

\begin{frame}[fragile]
  \frametitle{Formato de texto}
  \begin{lstlisting}[style=md]
Italica, con *asteriscos* o _guines bajos_.

Negrita, con doble **asteriscos** o __guiones bajos__.

Combinados con **asteriscos y _guiones bajos_**.

~~Texto tachado~~
  \end{lstlisting}
\end{frame}

\begin{frame}[fragile]
  \frametitle{Listas}
  \begin{columns}
    \column{0.45\textwidth}
      Simple
      \begin{lstlisting}[style=md]
* Item 1
* Item 2
* Item 3
      \end{lstlisting}\pause
    \column{0.45\textwidth}
      Numerada
      \begin{lstlisting}[style=md]
1. Item 1
2. Item 2
3. Item 3
      \end{lstlisting}
  \end{columns}\pause
  \vfill
  Listas anidadas (2 espacios en cada nivel)
    \begin{lstlisting}[style=md]
1. Item 1

  * Sub item 1
  * Sub item 2

2. Item 2
3. Item 3
\end{lstlisting}
\end{frame}

\begin{frame}[fragile]
  \frametitle{Enlaces}
  \begin{lstlisting}[style=md]
[texto](url)
  \end{lstlisting}\pause
  \vfill
  La inclusión de imágenes son un caso especial de enlaces.

  Poseen un signo de exclamación como prefijo.

  \begin{lstlisting}[style=md]
![alt text](url)
  \end{lstlisting}
\end{frame}

\begin{frame}[fragile]
  \frametitle{Código}
  \begin{columns}
    \column{0.45\textwidth}
    En línea (junto con el texto)
    \begin{lstlisting}[style=md]
texto y `código`
    \end{lstlisting}\pause
    \column{0.45\textwidth}
    En bloque separado
    \begin{lstlisting}[style=md]
```
código
```
    \end{lstlisting}\pause
  \end{columns}\pause
  \vfill
  Con resaltado de sintaxis
  \begin{lstlisting}[style=md]
```python
código
```

```suby
código
```
  \end{lstlisting}
\end{frame}

\section{Markdown \& \LaTeX}
\subsection{Ventajas y desventajas}
\begin{frame}
  \frametitle{\LaTeX}
  Sirve cuando necesitamos:
  \begin{itemize}
    \item Calidad tipográfica
    \item Que el diseño lo haga otro (en este caso, \LaTeX)
    \item Usar lenguaje específico (matemáticas, código, etc.)
  \end{itemize}\pause

  \vfill

  No sirve tanto si:
  \begin{itemize}
    \item Se tiene poco tiempo para aprender (último día de TP)
    \item Queremos trabajar sobre documentos existentes
  \end{itemize}
\end{frame}

\begin{frame}
  \frametitle{Markdown}
  Sirve para:
  \begin{itemize}
    \item Escribir documentos con formato sencillo
    \begin{itemize}
      \item Notas
      \item Mails
      \item Documentación web
    \end{itemize}
    \item Editar en varias plataformas (web, móvil)
  \end{itemize}\pause

  \vfill

  No sirve tanto si:
  \begin{itemize}
    \item El documento incluye gráficos, fórmulas matemáticas
    \item Se quiere dar estilos propios al documento rápidamente
  \end{itemize}
\end{frame}

\subsection{Compatibilidad}

\begin{frame}
  \frametitle{No son excluyentes!}
  \begin{enumerate}
    \item Escribimos rápido y usamos \texttt{markdown}
    \item Damos un estilo profesional con \LaTeX
  \end{enumerate}\pause
  \vfill
  \begin{enumerate}
    \item Tenemos un documento elaborado con \LaTeX
    \item Lo convertimos a \texttt{markdown} para subirlo, por ejemplo, a la web
  \end{enumerate}\pause
  \vfill
  Para lograr esto, usamos la herramienta \textbf{Pandoc}
\end{frame}

\section{Pandoc}

\begin{frame}
  \frametitle{La herramienta para conversión entre markups}
  \textbf{Pandoc} que permite convertir entre:
  \begin{itemize}
   \item \texttt{.md} (Markdown)
   \item \texttt{.tex} (\LaTeX)
   \item \texttt{.html}
   \item *Wiki
   \item \texttt{.epub} (e-books)
   \item muchos más...
  \end{itemize}
\end{frame}

\begin{frame}
  \frametitle{La herramienta para conversión entre markups}
  \emph{Realmente muchos más}
  \includegraphics[width=1\textwidth]{graficos/pandoc}
\end{frame}


\end{document}
