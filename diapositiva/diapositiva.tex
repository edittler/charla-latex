\documentclass[svgnames]{beamer}
\usetheme{Warsaw} % descomentar para cambiar el tema

\usepackage[utf8]{inputenc} % Indica cuál es la codificación de este archivo
\usepackage[spanish]{babel} % Indica el idioma en que está escrito el documento
\usepackage{fancybox}
\usepackage{graphicx}
\usepackage{amssymb} % Fuentes y símbolos adicionales (por AMS)
\usepackage{amsmath} % Mejoras a entornos matemáticos y extras (por AMS)
\usepackage{mathtools} % Correcciones a amsmath y funcionalidades extras

\usepackage{listings} % Permite ingresar código fuente de software

\lstset{ % Defino el formato de bloques de código fuente
  inputencoding=utf8, % Indico la codificación de los archivos de entrada
  %extendedchars=true, % Extiendo los caracteres
  % Escapeo caracteres especiales
  literate={¡}{{!`}}1 {¿}{{?`}}1 {á}{{\'a}}1 {é}{{\'e}}1 {í}{{\'i}}1 {ó}{{\'o}}1 {ú}{{\'u}}1 {ñ}{{\~n}}1,
  showstringspaces=false, % Indica si muestra los espacios dentro de strings
  numbers=left, % Posición en que se muestran los números de línea
  numberstyle=\tiny\color{gray}, % Estilo de los números de línea
  breaklines=true, % Se parten las líneas que exceden del ancho del documento
  frame=single, % Formato del marco del entorno
  backgroundcolor=\color{gray!5}, % Color de fondo
  basicstyle=\ttfamily\scriptsize, % Estilo de base (familia, tamaño, color)
}

\lstdefinestyle{latex}{
  tabsize=4,
  language=[LaTeX]TeX,
  keywordstyle=\color{DarkGreen}, % Estilo de las palabras reservadas
  stringstyle=\color{DarkBlue}, % Estilo de los strings
  commentstyle=\color{DarkGray}, % Estilo de los comentarios
}

\usepackage{hyperref}

% Beamer settings

\beamertemplatenavigationsymbolsempty

\AtBeginSection[]{
  \begin{frame}
    \vfill
    \centering
    \begin{beamercolorbox}[sep=8pt,center,shadow=true,rounded=true]{title}
      \usebeamerfont{title}\insertsectionhead\par%
    \end{beamercolorbox}
    \vfill
  \end{frame}
}

% Document Properties

\title{Herramientas libres para crear documentos de alta calidad}
\subtitle{Algo de {\LaTeX} y Markdown}
\author{Ezequiel ``Pampa'' Pérez Dittler}
\date{Semana LUGFI, Agosto de 2016}

\begin{document}

\frame{\titlepage}

\section{LUGFI}

\begin{frame}
  \frametitle{¿Qué es LUGFI?}

  Es el grupo de usuarios y desarrolladores de software libre y abierto de la Facultad de Ingeniería de la UBA

  Acrónimo recursivo:

  LUGFI Usa GNU/Linux en la Facultad de Ingeniería
\end{frame}

\section{Introducción a \LaTeX}

\begin{frame}
  \frametitle{¿Qué es \LaTeX?}
  \begin{itemize}
    \item Es un sistema de composición de textos
    \item Orientado a la creación de documentos con alta calidad tipográfica
    \item Muy popular en el entorno académico, especialmente entre matemáticos, físicos, químicos e informáticos
    \item Es un conjunto de macros de \TeX
    \item Es un lenguaje de maquetación (similar a HTML)
  \end{itemize}
\end{frame}

\begin{frame}
  \frametitle{¿Por qué usar \LaTeX?}
  \begin{itemize}
    \item Control flexible del contenido
    \item Fórmulas matemáticas de alta calidad
    \item Permite centrarse en el contenido, no en la presentación
    \item Extensible con multitud de paquetes
    \begin{itemize}
      \item La presentación se puede cambiar con multitud de plantillas
    \end{itemize}
  \end{itemize}
\end{frame}

\begin{frame}
  \frametitle{¿Para qué usar \LaTeX?}
  \begin{itemize}
    \item Trabajos prácticos
    \item Documentos con fórmulas matemáticas
    \item Tesis, papers
    \item Currículum Vitae
    \item Presentaciones
    \item Partituras (!)
  \end{itemize}
\end{frame}

\begin{frame}
  \frametitle{¿Cómo usar \LaTeX?}
  \framesubtitle{En forma local}
  Distribución \LaTeX
  \begin{itemize}
    \item Linux: TeX Live
    \item Windows: MiKTeX
    \item OSX (Mac): MacTex
  \end{itemize}

  Editor de texto
  \begin{itemize}
    \item Editor de texto plano favorito (por ejemplo, Bloc de notas)
    \item TeXStudio
    \item Texmaker
    \item LyX
  \end{itemize}
\end{frame}

\begin{frame}
  \frametitle{¿Cómo usar \LaTeX?}
  \framesubtitle{En línea}
  No requiere instalar nada, sólo se necesita un navegador de internet.
  \begin{itemize}
    \item Overleaf: www.overleaf.com
    \item ShareLaTeX: www.sharelatex.com
  \end{itemize}
\end{frame}

\section{Uso de \LaTeX}

\subsection{Conceptos básicos}
\begin{frame}
  \frametitle{¡Hola Mundo!}
  \lstinputlisting[style=latex]{hola-mundo/hola-mundo.tex}
\end{frame}

\begin{frame}
  \frametitle{Estructura del documento}
  \begin{columns}
    \column{0.45\textwidth}\centering
      \lstinputlisting[style=latex]{estructura/estructura.tex}
    \column{0.55\textwidth}\centering
      \begin{itemize}
        \item Preámbulo
        \begin{itemize}
          \item Inclusión y configuración de paquetes
          \item Configuración del documento
          \item Propiedades del documento
        \end{itemize}
        \item Contenido del documento
        \item Fin del documento
        \begin{itemize}
          \item Todo lo que se inserte aquí no tendrá efecto alguno tanto en la configuración como en el contenido
        \end{itemize}
      \end{itemize}
  \end{columns}
\end{frame}

\begin{frame}
  \frametitle{La codificación del caracteres del siglo XXI es... UTF-8!}
  \begin{columns}
    \column{0.45\textwidth}\centering
      \lstinputlisting[style=latex]{hola-mundo-utf/hola-mundo.tex}
    \column{0.55\textwidth}\centering
      \begin{itemize}
        \item Se incluyó el paquete \texttt{inputenc}
        \item Permite especificar la codificación del archivo fuente
        \item La opción \texttt{utf8} indica que la codificación es UTF-8
      \end{itemize}
  \end{columns}
\end{frame}

\begin{frame}
  \frametitle{¿La fuente debe ser de 12pt? ¿Hoja A4? ¿Tipo libro?}
  \centering
  \lstinputlisting[style=latex]{tamanios/hola-mundo.tex}
\end{frame}

\begin{frame}
  \frametitle{Más opciones para personalizar}
  \begin{columns}
    \column{0.45\textwidth}\centering
      Clases\\[.2cm]
      \begin{itemize}
        \item article
        \item book
        \item beamer
        \item IEEEtran
        \item exam
        \item y varios más...
      \end{itemize}
    \column{0.55\textwidth}\centering
      Opciones\\[.2cm]
      \begin{itemize}
        \item 10pt,11pt,12pt,. . .
        \item a4paper,letterpaper,b5paper,. . .
        \item twoside, oneside
        \item landscape
        \item onecolumn, twocolumn
        \item draft
        \item y varios más...
      \end{itemize}
  \end{columns}
\end{frame}

\begin{frame}
  \frametitle{Ordenando en secciones}
  \lstinputlisting[style=latex]{secciones/secciones.tex}
\end{frame}

\begin{frame}
  \frametitle{Agregando índices}
  \begin{itemize}
    \item \texttt{\textbackslash{}tableofcontents}
    \item \texttt{\textbackslash{}listoffigures}
    \item \texttt{\textbackslash{}listoftables}
    \item y algunos más...
  \end{itemize}
\end{frame}

\begin{frame}
  \frametitle{¿Que tal una carátula?}
  \lstinputlisting[style=latex]{caratula/hola-mundo.tex}
\end{frame}

\begin{frame}[fragile]
  \frametitle{¡Los títulos de índices y las fechas se ven en inglés!}
  \framesubtitle{El paquete \texttt{babel} viene al rescate}

  \begin{lstlisting}[style=latex]
\usepackage[spanish]{babel}
  \end{lstlisting}

  \begin{itemize}
    \item El paquete \texttt{babel} traduce titulos, fechas
    \item Habilita la separación silábica en el idioma indicado
    \item La opción \texttt{spanish} indica que el idioma del documento es el español
    \item Para usar \texttt{babel} o indicar idiomas puede requerir instalar paquetes en la distribución \LaTeX
  \end{itemize}
\end{frame}

\subsection{Fórmulas matemáticas}
\begin{frame}
  \frametitle{Escribiendo fórmulas matemáticas}
  \begin{itemize}
    \item Sintáxis específica
    \item Sólo hay que indicar que vamos a escribir fórmulas
    \begin{itemize}
     \item inline (junto con texto)
     \item en una nueva línea
    \end{itemize}
   \end{itemize}
\end{frame}

\begin{frame}
  \frametitle{Fórmulas matemáticas: Ejemplos}
  \only<1> {\lstinputlisting[style=latex]{matematicas/matematicas.tex}}
  \only<2> {Sean $ \alpha, \beta \in \mathbb{R} $\begin{equation*} \alpha + \beta = \beta + \alpha\end{equation*} }      
\end{frame}

\begin{frame}
  \frametitle{Fórmulas matemáticas: Ejemplos}
  Sumatorias
  {\lstinputlisting[style=latex]{matematicas/sumatoria.tex}}
  \begin{center}
    $\sum_{i=0}^{n} a_ix^i$   
  \end{center}\pause
  Integrales
  {\lstinputlisting[style=latex]{matematicas/integral.tex}}
  \begin{center}
    $\int_{a}^{b} x^2 dx$   
  \end{center}\pause
  Límites
  {\lstinputlisting[style=latex]{matematicas/limite.tex}}
  \begin{center}
    $\lim_{x\to\infty} f(x)$  
  \end{center}\pause

\end{frame}

\subsection{Gráficos}

\begin{frame}
  \frametitle{}
\end{frame}


\subsection{Código}

\begin{frame}
  \frametitle{Ventajas}
  \begin{itemize}
   \item Formato agradable a la vista
   \begin{itemize}
    \item Syntax highlighting
    \item Personalización de estilos (números de líneas, colores)
   \end{itemize}
   \item ¡Posibilidad de importar código fuente desde otros archivos!
  \end{itemize}
\end{frame}

\begin{frame}
  \frametitle{Incluyendo código en el documento}
  \begin{columns}
    \column{0.55\textwidth}\centering
      \lstinputlisting[style=latex, firstline=1, lastline=9]{codigo/importar-listings.tex}
    \column{0.45\textwidth}\centering
      \begin{itemize}
        \item Hay que importar el paquete \texttt{listings}
        \item Se puede personalizar el estilo del código (importar \texttt{color})
        \item También el tamaño de los tabs
        \item Y muchas cosas más...
      \end{itemize}
  \end{columns}
\end{frame}


\begin{frame}[fragile]
  \frametitle{Incluyendo código en el documento}
  \framesubtitle{Uso básico}
  \lstinputlisting[style=latex, firstline=13, lastline=15]{codigo/importar-listings.tex}
  Da como resultado:
  \begin{lstlisting}
  print("Hola mundo!")
  \end{lstlisting}\pause
  \emph{¡Lo estuvimos usando todo el tiempo!}
\end{frame}

\begin{frame}[fragile]
  \frametitle{Incluyendo código en el documento}
  \framesubtitle{Personalizando un poco más}
  \lstinputlisting[style=latex, firstline=17, lastline=23]{codigo/importar-listings.tex}
  Da como resultado:
\begin{lstlisting}[language=python]
  def main():
    """Saluda al mundo"""
    print("Hola mundo")
  
  main()
\end{lstlisting}
\end{frame}

\begin{frame}[fragile]
  \frametitle{Incluyendo código en el documento}
  \framesubtitle{Importando un archivo con código}
  Indicamos el nombre del archivo a mostrar:
  \lstinputlisting[style=latex, firstline=25, lastline=25]{codigo/importar-listings.tex}\pause
  Podemos incluir sólo algunas líneas:
  \lstinputlisting[style=latex, firstline=26, lastline=26]{codigo/importar-listings.tex}

\end{frame}


\end{document}
